\section{Case Study}
\label{sec:study}

In this section, we answer the second key question about the sampling-based
learning: {\em Can delaying scheduling the
remaining tasks till completing the sampled tasks hurt the
job performance, \eg completion time?
% \deadlineCS{or meeting deadline (SLO)}?
}
We answer this question through extensive simulation and testbed experiments.

Our approach is to design a generic scheduler, denoted as \gs, that schedules
jobs based on job runtime estimates to optimize a given performance metric, \eg
average job completion time (JCT).  We then plug into \gs different
prediction schemes to compare their effectiveness.
\rm{In particular, we compare four predictors:
(1) the sampling-based predictor \slearn,
(2) the distribution based predictor proposed in 3Sigma~\cite{3Sigma},
(3) a point estimate predictor,
and
(4) a LAS estimator.
(5) an Oracle estimator, which always predicts with 100\% accuracy.
We further compare with a FIFO-based scheme, where the scheduler
simply prioritizes jobs in the order of their arrival.
}

%\addaj{We do not evaluate against Kairos~\cite{kairos:socc2018} as the policy
%is inherently preemptive and doesn't provide same liveness guarantee.}

\input design

%\subsection{Simulation}
\input simulation

\if 0
\subsubsection{Evaluation}
\label{sec:study:testbed}
\input testbed
\fi

\if 0
\section{Case Study - II}
\label{sec:study2}

\subsection{Scheduling for Meeting Deadlines}
\label{sec:study2:design}
\fi
%\input deadline-design
