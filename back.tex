\section{Background and Related Work}
\label{sec:back}

In this section, we provide a brief background on the cluster scheduling problem,
review existing schedulers based on learning job runtime characteristics, and
discuss their weaknesses.

%\vspace{-0.1in}
\subsection{Cluster Scheduling Problem}
\label{sec:back:problem}

%Growing use of cloud for variety of applications, increase in shared usage of
%the cluster among multiple users and execution of diverse workloads in the same
%cluster due to cluster consolidation leads to challenge of appropriately
%scheduling jobs in order to best utilize the cluster. The notion of best
%utilization might vary. In some cases requirement is to ensure fairness accross
%users, in some other meeting deadlines for maximum number of jobs is most
%important. Whereas for some cases quickly completing maximum number of jobs is
%the desired goal. Also, in many cases minimizing the operational cost of the
%cluster is the primary goal and for achieving that it is important to minimize
%average job completion time. Last two situations leads to the cluster
%scheduling problem with goal of minimizing average job completion time. In this
%paper we focus on online cluster scheduling problem with goal of minimizing
%average job completion time.

%\questionaj{Why the first para here is not good? Is it not a way to tell that
%why scheduling is a problem?\\}
%Businesses ranging from Fortune-500 companies to small seed funded start-ups are
%increasingly relying on shared clusters for executing thier business critical
%jobs. The efficiency of cluster usage has a direct impact on business running cost.
%This makes efficient cluster scheduling and meeting desired scheduling
%objectives very crucial.

In both public and private clouds,
clusters are typicalling shared among multiple users to execute diverse jobs. Such
jobs typically arrive online and compete for shared resources. In order to best
utilize the cluster and to ensure that jobs also meet their service level
objectives (SLOs), efficient job scheduling is essential. Since jobs arrive online,
their runtime characteristics are not known a priori. This lack of information
makes it challenging for the scheduler to determine the right order for running
the jobs that maximizes resource utilization and/or meets application 
SLOs. Additionally, jobs have different SLOs. For some
meeting deadlines is important while for others faster completion or minimizing
the use of networks is more important. Such a diverse set of objectives pose
further challenges to effective job
scheduling~\cite{drf:nsdi11,jockey:eurosys2012, shufflewatcher, corral,
morpheus, delay:eurosys10, cdef:atc18}.

%\soccReviewEdit{A, B}{\paragraph{Job model.}
%\vspace{-0.1in}
\subsection{Job Model}
\label{sec:back:jobmodel}

We consider big-data compute clusters running data-parallel frameworks
such as Hadoop~\cite{hadoop:web}, Hive~\cite{hive:web},
Dryad~\cite{dryad:eurosys2007}, 
Scope~\cite{scope:2008},
and Spark~\cite{spark:web} that run simple MapReduce
jobs~\cite{mapreduce:osdi04} or more complex DAG-structured jobs, where each job
processes a large amount of data. Each job consists of one or multiple
stages, such as map or reduce, and each stage 
partitions the data into
manageable chunks and runs many parallel tasks, one for processing
each data chunk.  

\if 0
Examples of such big-data jobs may involve a single
phase of parallel compute tasks (\eg in the mapper stage of MapReduce)
or multiple phases of compute with dependency modeled by directed
acyclic graphs (DAGs)~\cite{mapreduceonline:nsdi2010, apache:tez,
  dryad:eurosys2007, dandelion:sosp2013}.  
In this paper, we focus on
a single phase of parallel tasks. Our job model % of a single phase of
parallel tasks is similar to that
of~\cite{borg,perforator:socc2016,IfYouAreLateDontBlameUs:socc14} and
forms the building block for generalized job scheduling strategies for
multi-phase (DAG) jobs.  }
\fi

\if 0
For our experiments and analysis in this paper, each job consists of a
single phase of parallel tasks. We assume one-phase model because (1)
the same model is assumed in previous work \cite{jamiasvu},
\cite{3Sigma} and \cite{stratus:socc2018}; (2) it is practical to
implement - in each phase the application manager submits to the YARN
scheduler a request for the tasks belonging to this phase, and the
YARN scheduler then decides how to schedule them. (3) Same as in
recent work on job scheduling~\cite{stratus:socc2018}, the traces used
in our experiment do not contain multi-phase information. 
We will discuss how to apply sampling-based learning 
to multi-phase jobs (DAGs) in \S\ref{subsec:dag}.
%  Note that scheduling (of multi-phase or single-phase jobs) is
%  orthogonal to and can be decoupled from learning job/task size.
}
\fi

%The increase in shared usage of the cluster among multiple users and
%execution of diverse applications in the same cluster leads to the challenge of
%appropriately scheduling jobs in order to best utilize the cluster resources.
%The notion of best utilization might vary from case to case. In some cases the
%requirement is to ensure fairness in resource sharing among users, in some
%other meeting deadlines for the maximum number of jobs is most important and
%for some minimizing the makespan is the desired goal.  Also, in many cases
%minimizing the operational cost of the cluster is the primary goal and for
%achieving that it is important to minimize average job completion time.  With
%such diverse needs scheduling 
%%Scheduling distributed jobs with the goal of best utilizing the cluster and
%%optimizing the scheduling goals like, fairness, minimizing average completion
%%time, minimizing intra-cluster commnunication etc., 
%itself is a challenging job \cite{drf:nsdi11, jockey:eurosys2012,
%shufflewatcher, corral}. Now with the growing use of cloud for executing a wide
%range of applications ranging from advance machine learning jobs to simple
%compute jobs in the same shared cluster has made this further challenging
%\cite{morpheus, 3Sigma}.

%\paragraph{Importance of learning runtime characteristics of jobs:} As
%discussed above, in the modern clusters online arriving jobs with diverse
%runtime characteristics are being executed in parallel.  Hence, the schedulers
%have very little information about runtime characteristics of jobs to be able
%to use optimal algorithms, like SJF, which are mostly offline in nature.  So to
%be able to use optimal algorithms or algorithms designed for offline settings
%in online settings it becomes very important to \textit{apriori} learn the
%runtime characteristics of the jobs to be scheduled.

%\vspace{-0.1in}
\subsection{Existing Learning-based Schedulers}
\label{sec:back:existing}

An effective way to tackle the challenges of cluster scheduling is to
learn and estimate runtime characteristics of pending jobs. If we can accurately
estimate jobs characteristics, we can leverage offline scheduling
algorithms that are known to be optimal, \eg Shortest Job First
for minimizing the average completion time.


\begin{table}[tp]
%\vspace{-0.05in}
\caption{Summary of selected previous work that use history-based learning techniques.}
\label{table:prevwork}
	%\questionaj{I have not mentioned all the prior works in table \ref{table:prevwork} which are in \S\ref{sec:back:existing}. The works I have mentioned in the table cover all varieties. I didn't add other things to keep the table succint. Do you think we should expand the table?\\}
%\commentaj{I am not mentioning prediction accuracy as they are differently measured in different papers and are on different traces. Comparison of those values are not very sensible.}
\centering
{\small
\vspace{-0.1in}
\begin{tabular}{|c|c|c|c|c|}
\hline
		\textbf{ Name} & \textbf{Property} & \textbf{Estimation} & \textbf{Learning} \\ %& \textbf{Cluster traces used}\\
			& \textbf{estimated} & \textbf{technique} & \textbf{frequency} \\ %& \textbf{(Private + Public)}\\
	 
%\hline
\hline	
	  %\textbf{Corral \cite{corral}} & Job runtime & Profiling models & Makes several impractical assum- &  1 + 1 \\
	  \textbf{Corral } & Job runtime & Offline model & On arrival \\ %& 1 + 1 \\
	  \textbf{\cite{corral}} &  & (not updated) & \\% &\\
\hline
	  \textbf{DCOSR} & Memory elasti- & Offline model & Scheduler \\ %& 0 + 0 \\
	  \textbf{\cite{DontCryOverSpilledRecords}}& city profile & (not updated) & dependent \\% &\\
\hline
	  \textbf{Jockey} & Job runtime & Offline & Periodic \\ %& 1 + 0 \\
	  \textbf{\cite{jockey:eurosys2012}}& & simulator  &  \\% &\\
\hline
	  \textbf{3Sigma} & Job runtime & Offline & On arrival \\ %& 3 + 1 \\
	  \textbf{\cite{3Sigma}} & history dist. & model &  \\% &\\

	%%%%% Morpheus was cut from text so we have removed it from table as well %%%%%	  
%\hline
%	  \textbf{Morpheus} & SLOs; Resource & Offline & On arrival \\ %& 1 + 0 \\
%	  \textbf{\cite{morpheus}} & requirements &  model   &  \\% &\\
%\hline
%	  \textbf{Tetrisched \cite{tetrisched}} &  &  &  & 0 + 1 \\
\hline
\end{tabular}
%\begin{tabular}{|c|c|c|c|c|c|}
%\hline
%		\textbf{ Name} & \textbf{Property} & \textbf{Estimation} & \textbf{Target} & \textbf{Learning}  & \textbf{Cluster traces used}\\
%			& \textbf{estimated} & \textbf{technique} & \textbf{job types} & \textbf{frequency} & \textbf{(Private + Public)}\\
%	 
%%\hline
%\hline	
%	  %\textbf{Corral \cite{corral}} & Job runtime & Profiling models & Makes several impractical assum- &  1 + 1 \\
%	  \textbf{Corral \cite{corral}} & Job runtime & Offline & Recurrent and & On arrival & 1 + 1 \\
%	  &  & static model & offline jobs &  &\\
%\hline
%	  \textbf{Jockey \cite{jockey:eurosys2012}} & Job runtime  & Offline & Recurrent jobs & Periodic & 1 + 0 \\
%	  &  & simulator & &  &\\
%\hline
%	  \textbf{3Sigma \cite{3Sigma}} & Job runtime & Distribution of job & All jobs & On arrival & 3 + 1 \\
%	  & & runtime history & &  &\\
%\hline
%	  \textbf{Morpheus \cite{morpheus}} & SLOs; Resource & Offline  & Recurrent jobs & On arrival & 1 + 0 \\
%	  & requirements &  static model &  &  &\\
%\hline
%	  \textbf{Don't cry \cite{DontCryOverSpilledRecords}} & Memory elasticity & Offline & Recurrent jobs & Scheduler & 0 + 0 \\
%	  & profile & static model &  & dependent &\\
%%\hline
%%	  \textbf{Tetrisched \cite{tetrisched}} &  &  &  & 0 + 1 \\
%\hline
%%\vspace{-0.2in}
%\end{tabular}
}
\vspace{-0.1in}
\end{table}

In deed the problem of learning runtime characteristics of jobs has been intensively
studied.  Cluster schedulers using various learning methods have been proposed
\cite{corral, morpheus, shufflewatcher, 3Sigma, tetrisched,
DontCryOverSpilledRecords, perforator:socc2016, Apollo:osdi2014, wsmith:IEEE98, stratus:socc2018, roughSetEstimation:IEEE:Shonali}.  
In essence, all previous learning schemes
are {\em history-based}, \ie they learn job characteristics by observations
made from the past job executions. In particular, existing history-based
learning approaches can be broadly categorized into the following groups, as
summarized in Table~\ref{table:prevwork}.



%\cite{DontCryOverSpilledRecords} attempt to come up with a resource assignment
%using offline models.
%\paragraph{Offline prediction or profiling.}

\paragraph{Learning offline models.}
\if 0
\commentaj{Sampling based approach is fundamentally different from online
updates.  Learning by sampling is in-principle same as offline learning for
scheduling all the non-sampling tasks as non-sampling tasks are not scheduled
till sampling is over. Whereas online updates act as an error correction
mechanism while scheduling in the remaining tasks, if any, of the job or to
estimate remaining execution time of running jobs while scheduling new jobs.\\}
Corral~\cite{corral} uses an offline predictor for estimating job running
times, which it acknowledges may not be highly accurate.
\fi
Corral's prediction model is designed with the primary assumption that most
jobs are recurring in nature. It also makes several additional assumptions,
such as the latency of each stage of a multi-stage job is directly proportional
to the amount of data processed by it, which do not always hold true~\cite{corral}.

DCOSR~\cite{DontCryOverSpilledRecords} predicts the memory usage for
data parallel compute jobs using an offline model built from a fixed
number of profile runs.  These profile runs are specific to the
framework and depend on the  framework's properties as well as the
hardware they run on. Any software update in the existing frameworks,
addition of new framework or hardware update will require an update in
profile.

%The Yarn capacity scheduler \cite{yarnCapacity:web} also uses offline modeling
%created by profiling data gathered from execution history. \todoaj{Find
%details about following - to do scheduling? modeling what? and how is the
%model used?}

%Tetris \cite{MultiResourcePackingForClusterSchedulers} also uses history
%for predicting job running times for recurring jobs. 
For analytics jobs that perform the same computation periodically on different sets
of data, Tetris~\cite{MultiResourcePackingForClusterSchedulers} takes
measurements from past executions of the job to estimate the requirements for
the current execution.

%{how to predict? -- aj: Paper doesn't provide any more details on it.}

%\questionaj{This paper uses 4 different ways for prediction. Two are plain
%hueristics. Third is for reccuring jobs and uses history. Another is for such
%jobs whose tasks arrive at different times and so they use measurements from
%first few tasks for the later task. The second one is not exactly sampling but
%in principle it is the same.  Do you think mentioning this paper might make
%confused about our novelity?}

%Another reservation based scheduler \cite{IfYouAreLateDontBlameUs:socc14}
%uses history baesd predictor. % Commenting this as no name for the scheduler or predictor, hard to cite and not very important. 

\paragraph{Learning offline models with periodic updates.}
Jockey \cite{jockey:eurosys2012} periodically characterizes job progress at
runtime, which along with a job's current resource allocation is used
{by an offline simulator to estimate the job's completion time
% . Based on   the estimated completion time, 
and update the job's resource allocation.}
%  {as input to an offline estimated random variable function. The output of the
% function is used to update the resource allocation of the job so that its utility
% is optimized.}
\rm{The running time estimates made by the simulator are based
on performance statistics extracted from one or more previous runs of the
recurring job.} Jockey's learning module relies on the periodicity of job
occurrences and cannot work with new jobs.

%Tetrisched \cite{tetrisched} leverages the estimated job running
%times \comment{how does it estimate?}  and deadlines provided by the
%reservation systems of scheduling frameworks like Hadoop YARN
%\comment{to plan ahead in deciding whether or not to wait for a busy preferred resource.} It
%also coordinates with the reservation system to re-evaluate the
%immediate-term scheduling plan for all pending jobs at every scheduling
%cycle. \comment{this sentence is strange. above has not talked about estimation?}
%It is based on assumption that most of the jobs will be
%reccuring in nature.

%Tetrisched \cite{tetrisched} works in conjunction with reservation systems of
%scheduling frameworks like Hadoop YARN and leverages information provided by
%them about job deadlines and estimated runtimes to plan ahead in deciding
%whether to wait for a busy preferred resource or work with less preferred
%options. Tetrisched coordinates with the reservation system to re-evaluate the
%immediate-term scheduling plan for all pending jobs every scheduling cycle.
%Tetrisched is based on assumption that most of the jobs will be reccuring in
%nature.

%Perforator \cite{perforator:socc2016} also leverages job structure and
%profiling to predict job runtimes.

\paragraph{Learning for non-recurrent jobs.}
JVuPredict~\cite{jamiasvu} uses a black-box approach to learn
job properties. Instead of using execution history from the exact same jobs,
it matches jobs on the basis of some common features such as application name,
job name, the user who owns the job, and the resource requested by the job.
Additionally, 
% instead of using just one mathematical metric for estimating
% running times from the distribution, 
it uses multiple metrics, such as rolling
average and median,
in estimating running times from the distribution.
In principle, this elaborate design of JVuPredict 
revokes the reliance on recurring jobs, but it still depends on historical data.

%\paragraph{Distribution based learning.}
3Sigma~\cite{3Sigma} 
%  shares the two ideas of
extends JVuPredict~\cite{jamiasvu}
% , \ie matching jobs on the basis of some common features and using multiple metrics,
% and hence is also applicable to non-recurring jobs.
by introducing a new idea on prediction: instead of using point
metrics to predict runtimes, it uses full distributions of relevant
runtime histories.
\if 0
3Sigma~\cite{3Sigma} differs from previous history-based learning in three ways.
(1) Instead of using point metrics to predict runtimes, it uses full
distributions of relevant runtime histories.
(2) Instead of using execution history from the
exact same jobs, it matches jobs on the basis of some common features such as
application name, job name, the user who owns the job, and the resource requested by the
job.
%
(3) Instead of using just one mathematical metric for
estimating running times from the distribution, it uses multiple metrics, such as
rolling average and median.
\fi
%  It maintains a running metric to pick a feature
%  value and metric pair that can be best estimator.
%It maintains a running metric which tells that
%which feature value and metric pair can be best estimator and always use the
%one with the highest score.
%  {Though in principle, this elaborate design of
% 3Sigma revokes the reliance on recurring jobs, it still depends on historical data.
However, since it is impractical to maintain precise distributions for each
feature value, it resorts to approximating distributions, which
compromises the benefits of having full distributions. 

In practice, the approach of matching similar jobs on 
one or a few common features often does not work well.  The
paper~\cite{3Sigma} reports that for over 23\% jobs, the prediction
error is at least 100\% and for many jobs it is off by an order of
magnitude. Such error in estimating a job's
runtime can affect all the jobs being scheduled around the same
time.

%As a consequence, it has to rely
%heavily on error correction mechanisms, \comment{which are expensive??? -- it is hard to say}.
%Each job also has a utility, as a funtion of runtime, assigned to it. The
%distribution along with the utility is used to  

\if 0
\commentaj{Paragraph on SLOs is not parallel with others. Probably this should
be merged with something else.\\}
\paragraph{Learning service level objectives (SLOs).} Morpheus \cite{morpheus}
uses historical data to learn user expectations. Using historical data it
attempts to codify implicit user expectations as explicit Service Level
Objectives (SLOs). Limitation of the learning module of Morpheus is that it
relies on periodicity in occurrence of jobs.
\fi

%Similarly other works \cite{IfYouAreLateDontBlameUs:socc14,
%MultiResourcePackingForClusterSchedulers} have also used history for prediction
%of job runtimes but they have relied heavily on recurrence of jobs or on
%the periodicity of jobs. Also, 3Sigma \cite{3Sigma}, Jockey
%\cite{jockey:eurosys2012}, Morpheus \cite{morpheus} have highlighted that how
%history can get outdated and might not be very reliable even for periodic jobs,
%we discuss this in details in \S\ref{sec:back:whatsWrong}.
%%The key idea and limitations of 



\subsection{Learning from History: Assumptions and Reality}
\label{sec:back:whatsWrong}

Predicting job runtime characteristics from history information relies on two
conditions to hold: (1) The jobs are recurring; (2) The performance of
the same {or similar} jobs
will remain consistent over time.  In the following, we argue why these
conditions may not be applicable to modern day clusters.

%One of the major assumptions history based predictors rely on is that most of
%the jobs are recurring.

%Different history-based predictors are designed in accordance with the distribution
%and behavior pattern of jobs in the cluster.
%%Though they incorporate different ways in their design to mitigate the harm
%%which will happen when such assumptions fail\cite{corral, jockey:eurosys2012,
%%morpheus, 3Sigma}.
%Such predictors will not perform in an expected way when there is a
%considerable change in the fraction of recurring jobs in the cluster.
%%Or overall the cluster property is not very similar to properties for which the
%%predictor was designed.
%Different cluster schedulers which use history-based prediction mechanisms,
%based on the requirement of their workload, incorporate different ways in
%their design to mitigate the harm which will happen when such assumptions
%fail\cite{corral, jockey:eurosys2012, morpheus, 3Sigma}. Such predictors might
%not perform in an expected way when there is a considerable change in the
%fraction of recurring jobs in the cluster or overall the cluster property is
%not very similar to properties for which the predictor was designed.

%Cluster schedulers which use these prediction mechanisms 
%appropriately build around exploiting this assumption in different ways
%\cite{jockey:eurosys2012, 3Sigma, morpheus, corral}.  As these schedulers are
%primarily exploiting the reccurance of jobs and with the 
%might not perform in expected way when there is considerable change in fraction
%of reccurring jobs in the cluster. 

%\noindent
%\vspace{-0.2in}
\paragraph{Condition 1: The jobs are recurring.}
Many previous work have acknowledged that not all jobs are
recurrent. For example,
the traces used in Corral \cite{corral} and Jockey
\cite{jockey:eurosys2012} show that only 40\% of the jobs are recurrent and
Morpheus \cite{morpheus} shows that 60\% jobs are recurrent.
%A trace analysis study
%\cite{workloadDiversity:atc18} has shown that there can be a huge variation in
%properties across traces from different clusters as well as among jobs in the
%same cluster.
%In 3Sigma \cite{3Sigma} performance gain measured for the same metric can vary
%upto 8$\times$ just by varying the trace.
%%In 3Sigma \cite{3Sigma},performance gain measured for the same metric can vary
%%from 1.5$\times$ to 8$\times$ just by varying the trace.
%Also, the number of jobs with >95\% error in predicted runtimes varies upto
%4$\times$ and number of jobs which have zero to negligible error varies upto
%3.5$\times$ across different traces.
%History based predictors, directly or indirectly, rely on some common
%assumptions.  However, many of these assumptions are not applicable for modern
%day's clusters.
%So, cluster schedulers have to provide mechanisms to mitigate
%effects of failure of these assumptions \cite{jockey:eurosys2012, 3Sigma,
%morpheus, corral}.  
%In this subsection, we discuss those assumption and their validity.

%\paragraph{Assumption 2: Characteristics of recurrent jobs are highly predictable.}
%Another property that history based predictors would ideally want to have is
%that characteristics of the recurrent jobs are highly predictable.
%\paragraph{Assumption 2: Runtime characteristics of newly arriving jobs can be
%predicted with high accuracy using historical data.\\}

%\noindent
%\vspace{-0.2in}
\paragraph{Condition 2: The performance of the same
{or similar} jobs will remain consistent over time.}
Previous works~\cite{3Sigma, morpheus, corral, jockey:eurosys2012} that exploited
history-based prediction have considered jobs in one of
the following two categories.
%\begin{itemize}
(1) {\em Recurring jobs}: A job is re-scheduled to run on newly arriving data;
(2) {\em Similar jobs:} A job has not been seen before but has some
attributes in common with some jobs executed in the past. The
attributes can be application name, job name, user of
the job, or amount of resources requested by the job.
%\end{itemize}
Many of the history-based approaches only predict for recurring jobs \cite{morpheus, corral,
jockey:eurosys2012}, while some others \cite{3Sigma, jamiasvu, stratus:socc2018, roughSetEstimation:IEEE:Shonali} work for both categories of jobs.

However, even the authors of history-based prediction schemes such as
3Sigma~\cite{3Sigma} and Morpheus~\cite{morpheus} strongly argued why
runtime properties of jobs, even with the same input, will not remain consistent
and will keep evolving.  The primary reason is that updates in cluster
hardware, application software, and user scripts to execute the cluster jobs
affect the job runtime characteristics.
{They found that}
in a large Microsoft production cluster, within a one-month period,
applications corresponding to more than 50\% of the recurring jobs were
updated. The source code changed by at least 10\% for applications
corresponding to 15-20\% of the jobs.  Additionally, over a one-year period,
the proportion of two different types of machines in the cluster
changed from 80/20 to 55/45. For a same production Spark job, there is
a 40\% difference between the running times observed on the two types of
machines~\cite{morpheus}.  

For these reasons, although the state-of-the-art history-based system 3Sigma~\cite{3Sigma} 
uses sophisticated prediction techniques,
% Machine Learningd also shown that
the predicted running times for more than 23\% of the jobs have at
least 100\% error, and for many the prediction is off by an order of
magnitude. In our analysis of two production cluster traces (see
Figure~\ref{fig:accuracy:trace_analysis_window}), we observed similar
levels of high variability in the runtime characteristics of the jobs with the same
attributes.

% \questionaj{The trace analysis will be disscused in
% \S\ref{sec:comparison:quantity}. Do you think there is any need to add more of it here?}

%Above observations make it quite evident that even in short duration runtimes
%can vary significantly for similar jobs also.\\
%The above observations make it clear that the assumption that a large fraction of
%jobs is recurrent is not very reliable.
%Additionally, characteristics of jobs
%can vary a lot from one cluster settings to others and hence the design of the
%scheduler needs to make those accomodations.
%\cite{jockey:eurosys2012} and \cite{morpheus} have been evaluated only on
%single trace.

\if 0
\paragraph{}
Although the previous history based predictors have acknowledged
the above discussed drawbacks, they did not provided a solution to tackle the
root cause of the limitation, \ie the assumption that
job runtime properties remains stationary over time \cite{jockey:eurosys2012, 3Sigma, morpheus, jamiasvu}.
%They atmost provide steps to mitigate the effects of
%the drawbacks \cite{jockey:eurosys2012, 3Sigma, morpheus, jamiasvu}.
\fi
