\section{Conclusions}
\label{sec:conc}

The ability to accurately estimate job runtime properties allows a cluster job
scheduler to effectively schedule jobs.  In this paper, we performed a
comparative study of task-sampling-based prediction and history-based
prediction commonly used in the current cluster job schedulers. Our study
answers two key questions: (1) Via quantitative, trace and experimental
analysis, we demonstrate that the task-sampling-based approach can predict job
runtime properties with much higher accuracy than history-based schemes.  (2)
Via extensive simulations and testbed experiments on a 150-node cluster in
Microsoft Azure of a generic cluster job scheduler, we show delaying
non-sampled tasks till completion of sampled tasks in sampling-based learning
can be more than compensated by the improved accuracy over the prior-art
history-based predictor, and as a result reduces the average JCT, by
1.28$\times$,
1.56$\times$, and 1.32$\times$ 
for three production cluster traces.
These results suggest task-sampling-based prediction offers a 
promising alternative to the history-based prediction in facilitating cluster
job scheduling. 

%  \deadlineCS{We also found that sampling and history based
%  approaches have similar accuracy when it comes to predicting the maximum task length.}
