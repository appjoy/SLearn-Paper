\section{Related Work}
\label{sec:related}

We already summarize previous work on history-based learning schemes
and job schedulers in \S\ref{sec:back:existing}. Here we summarize
additioal related work.

\textbf{Job scheduling:}
{Stratus~\cite{stratus:socc2018} is a cluster scheduler that focuses
on optimizing the dollar cost. It uses a runtime predictor
derived from JamiasVu~\cite{jamiasvu} which has feature selection mechanisms
similar to that used in 3Sigma~\cite{3Sigma}.}

The Kairos~\cite{kairos:socc2018} scheduler is built on the principle
  of Least Attained Service.  The policy is inherently preemptive and
  does not provide same liveness guarantee as \slearn with
  \gs. Further, \slearn is a learning based technique and Kairos
  is a non-learning approach.

%  We tried to access the soruce code of Kairos at the
%  url~\cite{kairosScheduler} provided in the paper, but we got a not found error.
%  We also wrote to the author however did not hear back till the time of writing
%  this article.  

\if 0
There have been much work on scheduling in analytic
systems and storage at scale by improving speculative tasks~\cite{late:osdi08,
mantri:OSDI2010, dolly:nsdi13}, improving locality~\cite{delay:eurosys10,
scarlett:eurosys11}, and end-point flexibility~\cite{sinbad:sigcomm13,
  pisces:osdi12}.
 
\fi
\textbf{Speculative scheduling:} Recent works
(\cite{creditscheduling:sigcomm17, trumpet:sigcomm16}) use the idea of
online requirement estimation for scheduling in datacenter.
%In Corral ~\cite{corral}, recurring big data %analytics jobs are scheduled using their history.\\

%  \textbf{Flow scheduling:} There have been a rich body of prior work on flow
%  scheduling to minimize flow completion time, both with prior information (\eg
%  PDQ~\cite{pdq:sigcomm12}, pFabric~\cite{pfabric:sigcomm13}) and without prior
%  information (\eg Fastpass~\cite{fastpass:sigcomm14},
%  PIAS~\cite{pias:hotnets14}, ~\cite{pias:nsdi15}). These problems are
%  orthogonal to the problem discussed in this paper. 

\if 0
{\textbf{Coflow scheduling:} Recent works,
  Graviton~\cite{graviton:hotcloud16} and Saath~\cite{jajooSaath}, in
  coflow scheduling have also exploited the spatial dimension to come
  up with efficient coflow scheduling algorithm.}
\fi
